%----------------------------------------
% Write your notes here
%----------------------------------------
\section{What is Hadoop?}
\begin{itemize}
	\item An abstraction for parallel data analysis
	\item consists of many subprojects
		\subitem - we will mostly focus on Map/Reduce
	\item deals with distributed data
	\item born out of open source web indexing, crawling, and searching software
	\item Map/Reduce revealed by Google in 2004
		\subitem - added to Hadoop, which is adopted by Yahoo! in 2006
\end{itemize}

\section{Why do we need Map/Reduce?}
\begin{itemize}
	\item There is still a lot of latency when dealing with a lot of data
	\item Read speed of commodity hard disk is about 1 TB/4hrs
	\item Using Hadoop, 1PB can be sorted in 16.5 hours! \href{bit.ly/petabytesort}{petabytesort}
\end{itemize}

\section{Map/Reduce} 
\begin{itemize}
	\item break into parts
	\item process in parallel
	\item combine results
\end{itemize} 

For example, if we wanted to count the number of occurences of each word in a book:
\begin{enumerate} 
	\item For every word on every page
	\item Map to (word, count) e.g. ("cat", 1) 
	\item Shuffle to collect all records with the same key (word)
		\subitem hash(val) = hashVal mod number of reducers
	\item Reduce results by adding count values for each word 
\end{enumerate}

To use Map/Reduce, you must specify the Map and Reduce steps

\section{Principles}
\begin{itemize}
	\item move code to data
	\item allow programs to scale transparently 
\end{itemize}

\section{Strengths}
\begin{itemize}
	\item batch, offline
	\item write-once read-many

\section{Introduction}
This is where your text goes.
If you're new to \LaTeX, check out Overleaf\footnote{\url{http://overleaf.com}}, an online \LaTeX~environment where you can edit and render your documents.
They also have a very useful \href{http://www.overleaf.com/help/18-how-do-i-use-overleaf}{getting started guide}.

Figure \ref{fig:example_figure} is an example of how to include an image.

\begin{figure}[ht]
  \begin{center}
    \includegraphics[width=0.5\textwidth]{figures/example_figure.png}
    \caption{
      This is how to include a figure.
      As long as you use pdflatex most file types (e.g., jpg, png, pdf) should work.}
    \label{fig:example_figure}
  \end{center}
\end{figure}

And here's some math:
\begin{equation}
  \int_{-\infty}^{+\infty} e^{-x^2} dx = \sqrt{\pi}
\end{equation}

You can also make numbered lists:
\begin{enumerate}
  \item Thing 1
  \item Thing 2
  \item etc.
\end{enumerate}

Or bulleted lists
\begin{itemize}
  \item Thing 1
  \item Thing 2
  \item etc.
\end{itemize}

It shouldn't get much more complicated than that.
