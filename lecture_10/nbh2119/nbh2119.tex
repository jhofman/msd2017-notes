

\section{History of Networks}

\begin{enumerate}
	\item 1930s - Relationships are first portrayed as a network. These initial networks had to be manually constructed by researchers. Due to the difficulties of collecting data, the datasets and resulting networks were very small.
	\item 1960s - Random graph theory develops. Networks can be randomly generated by adding nodes and connecting them with a given probability $p > \frac{((1 + \upvarepsilon) \ln n)}{n}$
	\item 1970s - Important research on social networks conducted by Mark Granovetter. He discovers that friends of friends almost always know each other. This results in triangles being formed far more frequently in real social networks than in randomly generated networks. He also studied the relationship between strong and weak ties. Strong ties tend to form triangles whereas weak ties tend to bridge various separate triangles. All of these observations are compared to the base case of a randomly generate graph. Triadic closure makes it hard to mathematically study graphs that represent reality.
	\item 1970s - Derek de Solla Price publishes an influential paper on the cumulative advantage effect. According to his research, popularity begets popularity, the rich get richer and this inequality increases over time.
	\item 1990s - Duncan Watts and Steven Strogatz develop tshe Watts and Strogatz model to bridge differences between random graphs and real networks. The small world model generates a regular graph and then rewires nodes to other random nodes with a probability of $p$. This model includes many real-world characteristics such as triadic closure while still allowing researchers to make mathematical statements about the network.
	\item 1990s - Researchers begin modeling the internet. Unlike social networks and the small world model, the models of the internet show lots of singular attention.
	\item 2000s - Political blogs are highly clustered according to political stance with very few interconnected links. This contributes to increased political polarization.
\end{enumerate}

\section{Types of Networks}
\begin{itemize}
	\item Social Networks
	\begin{itemize}
		\item People saying that they're friends and therefore connected
		\item Explicit network
		\item E.g. Facebook
	\end{itemize}
	\item Information Networks
	\begin{itemize}
		\item Pointing towards information
		\item Network of citations
		\item Explicit network
		\item E.g. The internet
	\end{itemize}
	\item Activity Networks
	\begin{itemize}
		\item Shows a connection between two people depending on an activity
		\item For example, two people have emailed each other or talked on the phone. It demonstrates a connection, but it is hard to define
		\item E.g. Email
	\end{itemize}
	\item Biological Networks
	\begin{itemize}
		\item Represent physical processes that are governed by very different forces
		\item E.g. Protein interactions
	\end{itemize}
	\item Geographical Networks
	\begin{itemize}
		\item Represents physical areas including maps and routing
		\item E.g. Road networks
	\end{itemize}
\end{itemize}
\noindent
Networks may be a combination of the above types or contain aspects of multiple types. For example, Facebook is a social network but has evolved to include aspects of an information network. By contrast, Twitter is a more pure information network. All of these very different data sources can all be abstracted to the same general thing.

\section{Representations of Networks}
There are many different levels of abstraction for representing networks and the attributes that you choose to include can drastically change the structure and subsequent interpretation of the network.
\begin{itemize}
	\item Directed vs undirected graphs. For example, being friends with someone vs following a page on Facebook
	\item Weighted edges
	\item Metadata such as node attributes or edge attributes
\end{itemize}

\section{Network Data Structures}
\begin{itemize}
	\item Array of tuples
	\begin{center}
		$\{[0,1],[0,6],[0,8],[1,4],[1,6],[1,9],[2,4],[2,6],[3,4],[3,5],[3,8],[4,5],[4,9],[7,8],[7,9]\}$
	\end{center}
	\begin{itemize}
		\item Simple to store.
		\item Difficult to compute with.
		\item For example, a query such as ``Who are the neighbors of node 4'' will take $O(E)$ time because there is no index and, consequently, the entire list must be scanned.
	\end{itemize}
	\item Adjacency matrix
	\begin{table}[ht]
	\centering
		\begin{tabular}{c|cccccccccc}
		 & 0 & 1 & 2 & 3 & 4 & 5 & 6 & 7 & 8 & 9\\
		\hline
		0 & 0 & 1 & 0 & 0 & 0 & 0 & 1 & 0 & 1 & 0\\
		1 & 1 & 0 & 0 & 0 & 1 & 0 & 1 & 0 & 0 & 1\\
		2 & 0 & 0 & 0 & 0 & 1 & 0 & 1 & 0 & 0 & 0\\
		3 & 0 & 0 & 0 & 0 & 1 & 1 & 0 & 0 & 1 & 0\\
		4 & 0 & 1 & 1 & 1 & 0 & 1 & 0 & 0 & 0 & 1\\
		5 & 0 & 0 & 0 & 1 & 1 & 0 & 0 & 0 & 0 & 0\\
		6 & 1 & 1 & 1 & 0 & 0 & 0 & 0 & 0 & 0 & 0\\
		7 & 0 & 0 & 0 & 0 & 0 & 0 & 0 & 0 & 1 & 1\\
		8 & 1 & 0 & 0 & 1 & 0 & 0 & 0 & 1 & 0 & 0\\
		9 & 0 & 1 & 0 & 0 & 1 & 0 & 0 & 1 & 0 & 0
		\end{tabular}
	\end{table}
	\begin{itemize}
		\item Each bit represents a connection between two nodes.
		\item Easy to compute linear algebra operations or check edges.
		\item Takes $O(C)$ time to lookup a specific connection between nodes.
		\item Takes $O(n)$ to check neighbors of a specific node, because you still have to scan down the entire column / row of the node.
		\item Symmetric about the diagonal if the graph is undirected.
		\item Often stored as a sparse matrix.
	\end{itemize}
	\item Adjacency list
	\begin{align*}
	0 \mapsto 1, 6, 8\\
	1 \mapsto 0, 4, 6, 9\\
	2 \mapsto 4, 6\\
	3 \mapsto 4, 5, 8\\
	4 \mapsto 1, 2, 3, 5, 9\\
	5 \mapsto 3,4\\
	6 \mapsto 0,1,2\\
	7 \mapsto 8,9\\
	8 \mapsto 0,3,7\\
	9 \mapsto 1,4,7
	\end{align*}
	\begin{itemize}
		\item Neighbors of a node are stored together.
		\item Good for graph traversal.
		\item Similar to a sparse matrix
	\end{itemize}
\end{itemize}

\section{Descriptive Statistics}
\begin{itemize}
	\item Degree: how many connections does a node have?
	\begin{itemize}
		\item Degree distributions
	\end{itemize}
	\item Path length: how long is the shortest path between two nodes?
	\begin{itemize}
		\item Breadth first search
	\end{itemize}
	\item Clustering: How many friends of friends are also friends?
	\begin{itemize}
		\item Triangle counting
	\end{itemize}
	\item Components: How many disconnected parts does the network have?
	\begin{itemize}
		\item Connected components
	\end{itemize}
\end{itemize}

\section{Working with Networks in R}
\textit{NetworkX in Python is another good library for working with networks}
\begin{itemize}
	\item Calculating Degree:
	\begin{itemize}
		\item Group by source node
		\item Count \# destination nodes $\rightarrow$ (source, edge)
		\item Group by degree
		\item Count \# source nodes
	\end{itemize}
	\item Calculating Path Length:
	\begin{itemize}
		\item Path.length.hist will plot the path length
		\item Easiest to pull out data from here
		\item Note: calculating the shortest path for all pairs is computationally expensive and slow
	\end{itemize}
\end{itemize}
