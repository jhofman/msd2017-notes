%----------------------------------------
% Write your notes here
%----------------------------------------

\section{On Networks}
\subsection{History}
\begin{itemize}
  \item '30s: Breaking news: Networks are a thing!
  \item '60s: Random graph theory: Erdos + Rengi ('59)
  \begin{itemize}
	\item thought of graphs as math, as objects to be studied
	\item high probability:  more clustered in one component
	\item low probability: more scattered across multiple components
  \end{itemize}
  \item '70s: Granovetter ('73): Clustering and weak ties
  \begin{itemize}
	\item The friends of my friends are often friends.
    \begin{figure}[ht]
    	\begin{center}
      	\includegraphics[width=0.2\textwidth]{figures/Figure_1_Granovetter.png}
      	\caption{
        	Granovetter: this is forbidden; it's impossible for A-C and A-B to be the case without B-C being a thing as well!}
    	\label{fig:figure1}
  		\end{center}
	\end{figure}
	\item Ties can be strong (triadic closure) or weak (bridges).
    \item I may not know too well the people who bridge me to other 		communities.
  \end{itemize}
  \item '70s - relatively recently: Cross-platform data outside of surveys for social networks isn't lying around, making it hard to study.
  \item '70s: de Solla Price ('65,'76): Cumulative advantage in citation networks -- in many other words:
	\begin{itemize}
		\item Uneven distribution of attention
		\item Popularity begets itself
		\item There are a few celebrities and a bunch of nobodies.
  	\end{itemize} 
  \item '90s: Watts + Strogatz ('98): Small-world networks
  	\begin{itemize}
		\item Randomly rewired edges of a regular network
		\item Bridged the gap between IRL and the completely random graph
		\item Featured short path lengths (ie. just a few hops from A to B), triadic closure, and bridging
  	\end{itemize} 
    \item '00s: Newman, Barabusi, Watts ('06): Empirical structure from actual data, ie. hairballs
    	\begin{itemize}
		\item Adamic + Glance ('05): Homophily        
		\item Warning: location of nodes (blogs) may be contrived.
		\item Favors the lowest-energy configuration: force-directed, springlike edges that collapse close-together nodes more densely together in parameter space.
  	\end{itemize} 
\end{itemize}


\subsection{Types of networks}
Networks are abstractions of different types of data. We can be handed social (think: Facebook), informational (think: the web, political blogs, citations), activity (think: email), biological, and even geographical (think: roads) networks. It's important not to lose sight of what's being abstracted to a network. 

Representations, ie. levels of abstraction
\begin{itemize}
	\item Undirected
    	\begin{figure}[ht]
    		\begin{center}
      	\includegraphics[width=0.2\textwidth]{figures/Undirected_friends.png}
      	\caption{
        	Bidirectional friendship (one would hope)}
    	\label{fig:figure2}
  			\end{center}
		\end{figure}
    \item Directed
    	\begin{figure}[ht]
    		\begin{center}
      	\includegraphics[width=0.15\textwidth]{figures/Directed_pg_followers.png}
      	\caption{
        	Directed network, eg. followers of a FB page}
    	\label{fig:figure3}
  			\end{center}
		\end{figure}
   \item Weighted (the old ARPAnet, the OG Internet, whose edge costs varied)
   \item Metadata: attributes of the nodes and edges themselves
   \item Ego networks: by changing the threshold for what constitutes a 'meaningful' interaction or relationship, we change what the network looks like. 'All my FB friends', for example, will be much denser than 'my carefully maintained relationships'.  
\end{itemize} 

\subsection{Data Structures of Networks}
\begin{itemize}
	\item Edge list: storage :) compute :(
    	\begin{itemize}
		\item Compute time $\propto$ number of edges     
		\item To check if edge is present, requires a big scan, linear through number of edges
  		\end{itemize} 
    \item Adjacency matrix: checking edges :) linear algebra :) 
    	\begin{itemize}
        \item Storage in a sparse matrix is more efficient
		\item The not as big scan: run down the row or column; but this gets less easy for directed graphs because the matrix for these aren't symmetric
        \item Compute time $\propto$ number of nodes  
  		\end{itemize} 
    \item Adjacency list: graph traversal :) 
    	\begin{itemize}
        \item Compute time $\propto$ average number of neighbors for all nodes
  		\end{itemize}     
\end{itemize}

Descriptive Stats of Networks: 
\begin{center}
    \begin{tabular}{| l | l | l |}
    \hline
    Stat & Definition & Associated algorithm \\ \hline
    Degree & \# connections a node has & Degree distributions (counting) \\ \hline
    Path length & Shortest path between 2 nodes & BFS \\ \hline
    Clustering & How many friends of friends are also friends? & Triangle counting \\ \hline
    Components & \# disconnected parts & Connected components \\
    \hline
    \end{tabular}
\end{center}

\section{Coding up Networks}
\begin{itemize}
	\item Computing degree distribution (ie. How many nodes have 1,2,etc. neighbors?)
    
    group by source
    
    count \# destination nodes $\rightarrow$ (source,degree)
    
    group by degree
    
    count \# source nodes
    
    \item Computing path length - sometimes we'll need to logscale to handle distributions for which magnitude comparisons make more sense (think celebrities again)
    
    BFS: every newly discovered node is at distance, or path length, of one more than the current maximum distance. All pairs' path length $\propto$ \# nodes x \# edges
    
    init nodes at $\infty$
    
    	\quad source dist 0
    
    	\quad curr boundary is source
    
    	\quad new boundary is empty
    
    explore non-empty boundary
    
    loop over all nodes in curr boundary
    
    explore each undiscovered neighbor
    
    	\quad dist $\leftarrow$ curr dist + 1
        
    	\quad add neighbor to boundary 
        
    curr boundary $\leftarrow$ next boundary
    
    terminate when no more 'next boundary'
    
	\item Computing connectedness
    
    init nodes at $\infty$
    
    pick random unreached node (until no unreached nodes left)
    
    run BFS from that node
    
    	\quad label everything that's reached as one component
        
	\item Counting number of mutual friends for every pair $\propto$ d$^{2}$; that is, the few celebrities among us will kill our computers.
    
    We can use an adjacency list - here, it's: i, j1, j2, and j3
    
    \begin{figure}[ht]
    		\begin{center}
      	\includegraphics[width=0.2\textwidth]{figures/mutual_friends.png}
      	\caption{
        	Counting mutual friends}
    	\label{fig:figure4}
  			\end{center}
		\end{figure}
    
    for each node:
    
    	\quad run over all pairs of its neighbors
        
        \quad increment count by 1
        
        \item Counting triangles: checking if the j's themselves are connected
        
        Now it makes more sense to use an adjacency matrix, which provides a better memory footprint - else, computation is hell.
        
        for each node:
        
        \quad run over all its neighbors
        
        increment node's count if neighbors are connected
        
        We can measure how clustered a network is, or how connected a person is, by taking the ratio of number of actual triangles over possible triangles. 
        
\end{itemize}
      



